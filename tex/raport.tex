\documentclass{article}
\usepackage{polski}
\usepackage{tgpagella}
\usepackage{hyperref}
\usepackage{graphicx}
\usepackage{caption}
\usepackage{subcaption}
\usepackage{amsmath}
\usepackage{booktabs}
\usepackage{rotating}
\usepackage{array}
\usepackage{multirow}
\usepackage{amsthm}
\usepackage{mdframed}
\usepackage[explicit]{titlesec}

\usepackage[
	left=1.5in, right=1.5in, top=.5in, bottom=.5in
]{geometry}

\hypersetup{
colorlinks=true,
urlcolor=blue
}

\graphicspath{
	{../graphics/}
}

\newmdtheoremenv{definition}{Definicja}


\author{Emil Olszewski, Artur Sadurski}
\date{\today}
\title{Prognozowanie popytu na energię elektryczną na amerykańskim rynku day-ahead. \\  \large Komputerowa analiza szeregów czasowych \\ \large Raport 2. }


\begin{document}
\maketitle

% ------------------- STRESZCZENIE --------------------------
\begin{abstract}
Poniższy raport przedstawia analizę szeregu czasowego opisującego obciążenie na sieci elektrycznej na podstawie danych z rynku amerykańskiego z przestrzeni dni od 01.01.2016 do 31.12.2017. 
\end{abstract}


% ---------------- RYNKI DAY-AHEAD --------------------------
\section{Rynki day-ahead} 

W przypadku energii elektrycznej do zawierania kontraktów kupna-sprzedaży pomiędzy spółkami energetycznymi a operatorami elektrowni i sieci dochodzi na rynku \textit{day-ahead}, który nie pozwala na ciągły handel między uczestnikami. Na taki rynek spływają oferty kupna i dostarczenia konkretnej ilości energii na \textbf{każdą godzinę następnego dnia}. Ceny na każdy z tych okresów wyznaczane są jako punkt przecięcia się \textbf{krzywej popytu} i \textbf{podaży}. 


% ----------------- SZEREGI ARMA ----------------------------
\section{Szeregi ARMA}
Głównym celem raportu będzie dopasowanie szeregu ARMA do danych, więc należy wpierw przypomnieć jego definicję. 

\begin{definition}

Szereg czasowy ${\left\lbrace X_t \right\rbrace}_{t \in Z}$ nazywamy szeregiem $\text{ARMA}(p, q)$ gdy da się go przedstawić jako

$$ X_t = \varepsilon_t + \sum_{i=0}^p \phi_i\,X_{t-i} + \sum_{i=0}^q \theta_i\,\varepsilon_{t-i} $$

gdzie współczynniki $\phi_i$ oraz $\theta_i$ to współczynniki modelu zaś $\varepsilon_i \sim \text{WN}(0, \sigma^2)$. 
\end{definition}


% ---------------- OPIS DANYCH -------------------------------
\section{Opis danych}
\href{https://www.kaggle.com/datasets/robikscube/hourly-energy-consumption}{Dane}, do których został dopasowany model zostały udostępnione w domenie publicznej. Przedstawiają one ilości energii elektrycznej na którą zostały zawarte kontrakty na rynku \textbf{PMJ} na każdą godzinie dni pomiędzy 01.01.2016 a 31.12.2017. Obejmują więc okres dwuletni. Horyzont czasowy specjalnie został dobrany tak aby można było zaobserwować różne sezonowości dotyczące cen energii elektrycznej, to jest \textbf{sezonowość dobową} (związaną z cyklem dzień-noc), \textbf{tygodniową} (dni robocze-weekend) oraz \textbf{roczną} (pory roku). 

\end{document}